

\documentclass[12pt, a4paper]{article}
   



\usepackage[utf8]{inputenc}

\usepackage[fleqn]{amsmath}
\usepackage{nccmath}
\usepackage{amsmath}
\usepackage{nccmath}

\usepackage{latexsym}  %\mathbb{R}

%**************

\usepackage{graphicx}
\usepackage{bezier}
\usepackage[spanish]{babel}
\usepackage{makeidx}
\usepackage{latexsym}
% acentos y cosas varias
\usepackage{epstopdf}
\usepackage{amssymb}
\usepackage{curves}
\usepackage{rotating}
\usepackage{amsmath}
\usepackage{epic}


%\usepackage{HVDASHLN}
%\usepackage{epsfig}


\usepackage[utf8]{inputenc}
\usepackage[T1]{fontenc}
\usepackage{lmodern} % load a font with all the characters


\usepackage{hyperref} 
\usepackage{xcolor}

%**************



%*************************************************************************************************************
%***  Formato del 
%documento                                                                      
%          ***
%*************************************************************************************************************

\textwidth       15 cm
\textheight      24 cm

\oddsidemargin    0 cm
\evensidemargin   0 cm

\topmargin        -1 cm


\pagestyle{empty}

%\parskip = 1.5 \baselineskip


%*************************************************************************************************************
%***  Comandos para distintos conjuntos de 
%numeros                                                         ***
%*************************************************************************************************************

\newcommand{\RR}{{\sf I}\hspace*{-1.5pt}{\sf R}}
\newcommand{\NN}{{\rm I}\hspace*{-1.5pt}{\rm N}}
\newcommand{\PP}{{\sf I}\hspace*{-1.5pt}{\sf P}}
\newcommand{\QQ}{\hspace*{2pt}{\sf I}\hspace*{-5pt}{\sf Q}}
\newcommand{\ZZ}{{\sf Z}\hspace*{-4.5pt}{\sf Z}}
\newcommand{\CC}{\hspace*{2.5pt}{\sf I}\hspace*{-5pt}{\rm C}}


%*************************************************************************************************************
%***  Comandos para subindices y 
%superindices                                                              ***
%*************************************************************************************************************
 
\newcommand{\Ri}{_{_R}}  
\newcommand{\Le}{_{_L}}

      

%*************************************************************************************************************
%***  Inicio de 
%documento                                                                      
%            ***
%*************************************************************************************************************



\begin{document}
\begin{center}
	{\bf 
	Ampliación de Matemáticas} \\
	{\bf Master Universitario en Sistemas Espaciales - 
	ETSIAE}                                                                                                             
	               \\
\end{center}

\vspace{1cm}


\pagestyle{empty}

\noindent
{\bf \Large Weekly milestones. First semester.  
}                                                                               
     \\

\vspace{-0.5cm}

\begin{itemize}
	
	
\vspace{1cm}
\item {\bf Milestone 0 : First steps with the NumericalHUB }



Install the Microsoft Visual Studio and the Intel Fortran compiler by following 
the detailed installation manual:   
  \color{red} 
 \begin{center} 
        \href{https://www.amazon.es/s?k=hernandez+rapado+moreno+visual&__mk_es_ES=AMAZON&ref=nb_sb_noss}
        {Book: Programming with 
          Visual Studio}
  \end{center} 
\color{black}

 
A pdf  file   of this book can be found in: 
\color{red} 
\begin{center}

  \href{https://github.com/jahrWork/Visual-Studio-projects}{PDF 
  file:Programming with 
  Visual Studio}

\end{center}
\color{black}



Download and open the NumericalHUB by following the instructions given in: 
\color{red} 
\begin{center}

  \href{https://github.com/jahrWork/NumericalHUB}{Software: NumericalHUB}

\end{center}
\color{black}                

Open the \texttt{main\_NumericalHUB.f90} file and follow the folder structure 
and menu. Focus on the Cauchy Problem which is explained in the book: 

\color{red} 
\begin{center}
  \href{https://www.amazon.es/s?k=modern+fortran+hernandez&__mk_es_ES=%C3%85M%C3%85%C5%BD%C3%95%C3%91&ref=nb_sb_noss}
  {Book: How to learn applied mathematics through modern Fortran}
  
   \href{https://github.com/jahrWork/NumericalHUB}
    {PDF file: How to learn applied mathematics through modern Fortran}
\end{center}
\color{black}
 
Once the \texttt{NumericalHUB} is open,  revise the fisrt order and second 
order examples to understand how to reduce any ordinary differential equation 
to a system of first order differential equations. 

Write a Python modules for the following milestones and compare the results 
with those obtained with the Python library \texttt{scipy}


\item {\bf Milestone 1 : 
Prototypes to integrate orbits without functions.
  } 

\begin{enumerate} 
\item Write a script to integrate Kepler orbits with an Euler method. 
\item Write a script to integrate Kepler orbits with a Crank-Nicolson method. 
\item Write a script to integrate Kepler orbits with a Runge--Kutta fourth order.
\item Change time step and plot orbits. discuss results. 
 \end{enumerate} 



%******************************************************
\newpage 
\item {\bf Milestone 2 : Prototypes to integrate orbits with functions. } 

\begin{enumerate} 
\item Write a function called Euler to integrate one step. 
The function  $F(U,t)$ of the Cauchy problem should be input argument.
\item Write a function called Crank\_Nicolson to integrate one step.
\item Write a function called RK4 to integrate one step.
\item Write a function called Inverse\_Euler to integrate one step.
\item Write a function to integrate a Cauchy problem. Temporal scheme, initial condition and 
the function $F(U,t)$ of the Cauchy problem should be input arguments.
\item Write a function to express the force of the Kepler movement.
Put emphasis on the way the function of the Cauchy problem is written: 

$ F = [ \dot r , -r/|r|^3 ] $ where $ r, \dot r \in \mathbb{R}^2 $ 

\item Integrate a Kepler with these latter schemes and explain the results.
\item Increase and decrease the time step and explained the results. 
%\item Compare  these results with those  obtained with the library  \textt{scipy}. 
 
\end{enumerate} 

%******************************************************


\item {\bf Milestone 3 : Error estimation of numerical solutions.} 


\begin{enumerate}

\item Write a function to evaluate errors of numerical integration by means of Richardson 
extrapolation. 
This function should be based on the Cauchy problem solution implemented in milestone 2. 
\item Numerical error or different temporal schemes:  
Euler, Inverse Euler, Crank Nicolson 
and  fourth order Runge Kutta method.  
\item Write a function to evaluate the convergence rate of different temporal schemes. 
\item Convergence rate of the different methods with the time step.   

\end{enumerate} 


%******************************************************
\item {\bf Milestone 4 : Linear problems. Regions of absolute stability.}  
  \begin{enumerate}
\item  Integrate the linear oscillator $ \ddot x+ x =0$ with some initial conditions.  
       Use Euler, Inverse Euler, Leap--Frog, Crank--Nicolson and fourth order Runge Kutta
        method.  
\item Regions of absolute stability of the above methods. 
\item Explain the numerical results based on regions of absolute stability.
\end{enumerate}
 

%******************************************************
\newpage
\item {\bf Milestone 5 : N  body problem.}    

 \begin{enumerate} 
    
     \item Write a function to integrate the N body problem.  
     \item Simulate an example and discuss the results. 
 \end{enumerate}


%***************************************************

\item {\bf Milestone 6 : Lagrange points and their stability. }                                                      




\begin{enumerate}
\item Write a high order embedded Runge-Kutta method. 
\item Write function to simulate  the  circular restricted three body 
problem.
\item Determination of the Lagrange points $ F(U) = 0$. 
\item Stability of the Lagrange points: $ L_1, L_2 ,L_3,  L_4, L_5 $.
\item Orbits around the Lagrange points by means of different temporal schemes.

 \end{enumerate}
 
 
%******************************************************

\item {\bf Milestone 7 : Orbits of the circular restricted three body 
    problem.}    

 \begin{enumerate} 
    
     \item Investigate existing temporal schemes to integrate Cauchy problems. 
     Fortran and Python programming codes.  
     Use ODE integrators of the library scipy (odeint, solve\_ivp). 
     LSODA, ODEPACK, ODEX.  
     \item Integrate Arenstorf's periodic orbit. Compare results among GBS, RK and 
     AM methods.  
     \item Stability of Lyapunov orbits.  Shampine and Gordon orbits.
     \item Error tolerance and computational effort. 
     \item Mixing Fortran and Python: calling Fortran from Python and viceversa.
     \item Parallel programming with GPUs. N-body problem. 
     \item Standalone Python codes.  
     \item Automatic testing with GitHub.
     \item Mixing Fortran, C++ and Python in one executable file.
     \item Python script for GMAT. 
 \end{enumerate}






 
 
\end{itemize}

\end{document} 















\end{enumerate}


 
 
 
 
 
 
 
 
 
 
 
 
 
 
 
 
 
 
 
 
 
 
 
 
 
 
 
 
 
 
 
 
 
 
 
 
\newpage


%***********************************************
    \item {\bf Hito 5 : }   Funciones y gráficas. 
 
 \begin{enumerate}    
 \item Escribir un programa que dibuje las siguientes funciones: 
 
 
 
 \begin{fleqn}
  \[ f(x)  = \sqrt{ 1 - ( |x| -1 )^2 } \]
  
  \[ f(x) = \arccos(1 - | x | )  + \pi \]
  \[ f(x) =  \frac{ \cos( 70 x ) }{ (1 + x^2 )^2  }\]
 \[ f(x) = \sqrt{ \cos(x) }  \]
  \[ f(x) =  8 x \sin( x^2 + 3)  \]
 \[ f(x) = \prod_{k=0} ^M ( x - k/M )  \]
 \end{fleqn}
 
\item Dibujar el lugar geométrico de las raíces de: 

\[ x^2 + 2 \zeta \omega_0 x + \omega_0 ^2 = 0,\]

con $ \omega_0 = 1 $ para todo $ \zeta \ge 0$. 
 
\end{enumerate}



%***********************************************
    \item {\bf Hito 6 :}  Aproximar  mediante un desarrollo en serie de 
    potencias de la forma
    \[  f(x) = \sum_{k=0} ^M a_k \  x^k, \qquad a_k = \frac{  f^{(k)} (0)  }{ 
    k! }  \]  
    las  siguientes funciones: 
   
    \begin{fleqn}
  \[   f(x)  = \cos(x)  \]
    \[ f(x) = \cosh(x) \]
 \end{fleqn}
    
 
 
 
 \begin{enumerate}
   \item Calcular los desarrollos anteriores para un valor de $ M $ genérico.  
   \item  Dibujar la función a aproximar junto con las parmétricas que 
   constituyen los desarrollos anteriores con $ M=1, 2, 3, 4, 5. $
   \item Discutir los resultados obtenidos.
 \end{enumerate}
 
 

%***********************************************
    \item {\bf Hito 7 :}   Ceros de funciones. 
    
  Dada una función genérica $ f(x) $ escribir una programa que permita calcular 
  los ceros mediante: 
 
 
 \begin{enumerate}
   \item Algoritmo de la bisección. 
   \item Algoritmo de Newton. 
   \item Hacer aplicación a una función ejemplo y dibujar el resultado 
   obtenido. 
 \end{enumerate}
 
\end{itemize}

\end{document} 

